% ------------------------------------------------
% LaTeX Template for ML/AI Research Paper
% ------------------------------------------------

\documentclass[11pt]{article}

% ------------------------------------------------
% Packages
% ------------------------------------------------
\usepackage{amsmath, amssymb}
\usepackage{graphicx}
\usepackage{hyperref}
\usepackage{natbib}          % For bibliography
\usepackage{geometry}        % For adjusting margins
\usepackage{times}           % For Times New Roman font
\usepackage{float}           % For controlling float positions
\usepackage{booktabs}        % For professional-looking tables
\usepackage{caption}
\usepackage{subcaption}
\usepackage{color}
\usepackage{tikz}

% ------------------------------------------------
% Page Setup
% ------------------------------------------------
\geometry{margin=1in}

% ------------------------------------------------
% Title and Author Information
% ------------------------------------------------
\title{CNN Improved Transformer without positional Encoding with memory}
\author{
  Bastian Apportin\thanks{Corresponding author: \texttt{bastian.apportin@joomo.de}} \\
  \normalsize Joomo GmbH \\
%  \and
%  Co-author Name \\
%  \normalsize Co-author Affiliation \\
%  % Add more authors if necessary
}
\date{\today}

% ------------------------------------------------
% Document Begins
% ------------------------------------------------
\begin{document}

\maketitle

% ------------------------------------------------
% Abstract
% ------------------------------------------------
\begin{abstract}
% Provide a concise summary of the entire paper (150-250 words).
% Include background, methods, results, and conclusions.
The success of LLM's is mainly based on the astonishing capabilities of the Transformer Architecture. 
Which does a grat job in catching long range dependencies.

The positional encoding used in Transformers solves the problem of conbtext awarness, but limits the distance a transformer can look. What we want is aktually 
to keep the context awarness of the architecture, like distance memories coming back, influencing your current thoughts.

We've tried a lot of different aproaches like ALiBi or placing cnn's in the key/query calculations, but none of them worked like we expected. All the limited
experiments we did worked significantly better when using RoPE like positin encodings. So maybe, we don't need to remove them.

The RoPE embedding can be seen as an offset plus content adressing. The offset is the rotation and the mixture of the frequencies can adress sharper or broader position
back from the current position. 



\end{abstract}
% Optionally, you can add keywords
% \vspace{0.5cm}
% \noindent\textbf{Keywords:} Keyword1, Keyword2, Keyword3

% ------------------------------------------------
% Main Content
% ------------------------------------------------

% 1. Introduction
\section{Introduction}
% Introduce the topic, establish the importance, and state the main contributions.

The transformer architecture is self attention as its core. The worst and best part of that is, that there is no local context. Therefore 
most Transformer architectures apply some kind of positional encoding. This positional encoding solves the problem of the missing local context
but introduces the problem that for long distances the 'context' or distance matters as well. Therefore the view of site can not be infinite, which 
effectivly limits memorizing things by that mechanism.

The RoPE embedding can be seen as an offset plus content adressing. The offset is the rotation and the mixture of the frequencies can adress sharper or broader position ranges
back from the current position and the content are the original values of the queries and keys. 

% - Context

When we have positional adressing an idea came to mind, the memory mapped IO, where registers to access secondary hardware are mapped to main memory. One would write the block
id of a harddrive to a memory position and then the content of that block will be mapped to a specific position in memory.



% - Problem Statement
% - Objectives
% - Contributions
% - Structure of the paper



% 2. Methodology
\section{Methodology}

- use llama 3.2 1b
- explain using rotatry embeddings with fixed 'current position'
- map lsh like storage to defined position 0
- search by unrotated query
- provide a way to place memories in that storage
- use questions that indicate that the memories where used 

\begin{figure}[h]
\centering
\includegraphics[scale=0.2]{JoMoTra.jpg}
\caption{Architecture}
\label{fig:Architecture}
\end{figure}

% - Theoretical Background
% - Model Architecture
%\begin{tikzpicture}
%\draw (0,0) -- (4,0) -- (3,2) -- (1,2) -- (0,0);
%\end{tikzpicture}
% - Algorithms
% - Assumptions
% - Implementation Details

% 3. Experiments
\section{Experiments}

- define set of memories to use
- define set of questions, that use this memories.

% Describe how you tested your model and ensure reproducibility.
% - Datasets
% - Experimental Setup
% - Hyperparameters
%parameters: 92793376, conv1d(16,32,kernel=7) conv1d(32,512,kernel=7) 32*block_layer(hidden_size=512 ) conv1d(512,258,kernel=3)
%parameters: 400096, conv1d(16,32,kernel=7) conv1d(32,32,kernel=7) 32*block_layer(hidden_size=32 ) conv1d(32,258,kernel=3) loss=3.6
% - Evaluation Metrics

% 4. Results
\section{Results}
% Present the findings of your experiments.
% - Quantitative Results (Use tables and figures)
% - Qualitative Results
% - Comparisons
% - Statistical Analysis

% Example of including a figure
% \begin{figure}[H]
%     \centering
%     \includegraphics[width=0.7\textwidth]{path/to/your/image.png}
%     \caption{Caption of the figure.}
%     \label{fig:label}
% \end{figure}

% Example of including a table
% \begin{table}[H]
%     \centering
%     \caption{Caption of the table.}
%     \label{tab:label}
%     \begin{tabular}{lcc}
%         \toprule
%         Header1 & Header2 & Header3 \\
%         \midrule
%         Row1 & Data & Data \\
%         Row2 & Data & Data \\
%         \bottomrule
%     \end{tabular}
% \end{table}

% 5. Discussion
\section{Discussion}
% Interpret the results and discuss their implications.
% - Insights
% - Limitations
% - Practical Implications
% - Theoretical Implications

% 6. Conclusion
\section{Conclusion}
% Summarize the research and suggest future directions.

We successfully removed the positional encodings from the transformer architecture by adding cnn driven local context and thereby 
offered a way to have infinite context size. This can lead to large language models that do actually have a memory. 



% - Summary of findings
% - Significance
% - Future Work

% 7. Related Work
\section{Related Work}
% Situate your research within the existing body of work.
% - Literature Review
% - Critical Analysis
% - Positioning

% 8. Acknowledgments
\section*{Acknowledgments}
% Credit those who contributed to the research but are not listed as authors.
% - Funding Sources
% - Collaborations

% ------------------------------------------------
% References
% ------------------------------------------------
\bibliographystyle{plainnat}
\bibliography{references}

% ------------------------------------------------
% Appendices (Optional)
% ------------------------------------------------
\appendix

\section{Appendix Title}
% Include supplementary material that supports the paper but is too detailed for the main text.
% - Mathematical Proofs
% - Additional Figures or Tables
% - Code Listings

% Example of an equation
% \begin{equation}
%     E = mc^2
% \end{equation}

% ------------------------------------------------
% Document Ends
% ------------------------------------------------
\end{document}
